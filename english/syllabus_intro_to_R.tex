% Options for packages loaded elsewhere
\PassOptionsToPackage{unicode}{hyperref}
\PassOptionsToPackage{hyphens}{url}
\PassOptionsToPackage{dvipsnames,svgnames,x11names}{xcolor}
%
\documentclass[
  letterpaper,
  DIV=11,
  numbers=noendperiod]{scrartcl}

\usepackage{amsmath,amssymb}
\usepackage{iftex}
\ifPDFTeX
  \usepackage[T1]{fontenc}
  \usepackage[utf8]{inputenc}
  \usepackage{textcomp} % provide euro and other symbols
\else % if luatex or xetex
  \usepackage{unicode-math}
  \defaultfontfeatures{Scale=MatchLowercase}
  \defaultfontfeatures[\rmfamily]{Ligatures=TeX,Scale=1}
\fi
\usepackage{lmodern}
\ifPDFTeX\else  
    % xetex/luatex font selection
\fi
% Use upquote if available, for straight quotes in verbatim environments
\IfFileExists{upquote.sty}{\usepackage{upquote}}{}
\IfFileExists{microtype.sty}{% use microtype if available
  \usepackage[]{microtype}
  \UseMicrotypeSet[protrusion]{basicmath} % disable protrusion for tt fonts
}{}
\makeatletter
\@ifundefined{KOMAClassName}{% if non-KOMA class
  \IfFileExists{parskip.sty}{%
    \usepackage{parskip}
  }{% else
    \setlength{\parindent}{0pt}
    \setlength{\parskip}{6pt plus 2pt minus 1pt}}
}{% if KOMA class
  \KOMAoptions{parskip=half}}
\makeatother
\usepackage{xcolor}
\setlength{\emergencystretch}{3em} % prevent overfull lines
\setcounter{secnumdepth}{5}
% Make \paragraph and \subparagraph free-standing
\ifx\paragraph\undefined\else
  \let\oldparagraph\paragraph
  \renewcommand{\paragraph}[1]{\oldparagraph{#1}\mbox{}}
\fi
\ifx\subparagraph\undefined\else
  \let\oldsubparagraph\subparagraph
  \renewcommand{\subparagraph}[1]{\oldsubparagraph{#1}\mbox{}}
\fi


\providecommand{\tightlist}{%
  \setlength{\itemsep}{0pt}\setlength{\parskip}{0pt}}\usepackage{longtable,booktabs,array}
\usepackage{calc} % for calculating minipage widths
% Correct order of tables after \paragraph or \subparagraph
\usepackage{etoolbox}
\makeatletter
\patchcmd\longtable{\par}{\if@noskipsec\mbox{}\fi\par}{}{}
\makeatother
% Allow footnotes in longtable head/foot
\IfFileExists{footnotehyper.sty}{\usepackage{footnotehyper}}{\usepackage{footnote}}
\makesavenoteenv{longtable}
\usepackage{graphicx}
\makeatletter
\def\maxwidth{\ifdim\Gin@nat@width>\linewidth\linewidth\else\Gin@nat@width\fi}
\def\maxheight{\ifdim\Gin@nat@height>\textheight\textheight\else\Gin@nat@height\fi}
\makeatother
% Scale images if necessary, so that they will not overflow the page
% margins by default, and it is still possible to overwrite the defaults
% using explicit options in \includegraphics[width, height, ...]{}
\setkeys{Gin}{width=\maxwidth,height=\maxheight,keepaspectratio}
% Set default figure placement to htbp
\makeatletter
\def\fps@figure{htbp}
\makeatother
% definitions for citeproc citations
\NewDocumentCommand\citeproctext{}{}
\NewDocumentCommand\citeproc{mm}{%
  \begingroup\def\citeproctext{#2}\cite{#1}\endgroup}
\makeatletter
 % allow citations to break across lines
 \let\@cite@ofmt\@firstofone
 % avoid brackets around text for \cite:
 \def\@biblabel#1{}
 \def\@cite#1#2{{#1\if@tempswa , #2\fi}}
\makeatother
\newlength{\cslhangindent}
\setlength{\cslhangindent}{1.5em}
\newlength{\csllabelwidth}
\setlength{\csllabelwidth}{3em}
\newenvironment{CSLReferences}[2] % #1 hanging-indent, #2 entry-spacing
 {\begin{list}{}{%
  \setlength{\itemindent}{0pt}
  \setlength{\leftmargin}{0pt}
  \setlength{\parsep}{0pt}
  % turn on hanging indent if param 1 is 1
  \ifodd #1
   \setlength{\leftmargin}{\cslhangindent}
   \setlength{\itemindent}{-1\cslhangindent}
  \fi
  % set entry spacing
  \setlength{\itemsep}{#2\baselineskip}}}
 {\end{list}}
\usepackage{calc}
\newcommand{\CSLBlock}[1]{\hfill\break\parbox[t]{\linewidth}{\strut\ignorespaces#1\strut}}
\newcommand{\CSLLeftMargin}[1]{\parbox[t]{\csllabelwidth}{\strut#1\strut}}
\newcommand{\CSLRightInline}[1]{\parbox[t]{\linewidth - \csllabelwidth}{\strut#1\strut}}
\newcommand{\CSLIndent}[1]{\hspace{\cslhangindent}#1}

\KOMAoption{captions}{tableheading}
\makeatletter
\@ifpackageloaded{caption}{}{\usepackage{caption}}
\AtBeginDocument{%
\ifdefined\contentsname
  \renewcommand*\contentsname{Table of contents}
\else
  \newcommand\contentsname{Table of contents}
\fi
\ifdefined\listfigurename
  \renewcommand*\listfigurename{List of Figures}
\else
  \newcommand\listfigurename{List of Figures}
\fi
\ifdefined\listtablename
  \renewcommand*\listtablename{List of Tables}
\else
  \newcommand\listtablename{List of Tables}
\fi
\ifdefined\figurename
  \renewcommand*\figurename{Figure}
\else
  \newcommand\figurename{Figure}
\fi
\ifdefined\tablename
  \renewcommand*\tablename{Table}
\else
  \newcommand\tablename{Table}
\fi
}
\@ifpackageloaded{float}{}{\usepackage{float}}
\floatstyle{ruled}
\@ifundefined{c@chapter}{\newfloat{codelisting}{h}{lop}}{\newfloat{codelisting}{h}{lop}[chapter]}
\floatname{codelisting}{Listing}
\newcommand*\listoflistings{\listof{codelisting}{List of Listings}}
\makeatother
\makeatletter
\makeatother
\makeatletter
\@ifpackageloaded{caption}{}{\usepackage{caption}}
\@ifpackageloaded{subcaption}{}{\usepackage{subcaption}}
\makeatother
\ifLuaTeX
  \usepackage{selnolig}  % disable illegal ligatures
\fi
\usepackage{bookmark}

\IfFileExists{xurl.sty}{\usepackage{xurl}}{} % add URL line breaks if available
\urlstyle{same} % disable monospaced font for URLs
\hypersetup{
  pdftitle={Introduction to R},
  pdfauthor={Laboratorio de Investigación para el Desarrollo del Ecuador},
  colorlinks=true,
  linkcolor={blue},
  filecolor={Maroon},
  citecolor={Blue},
  urlcolor={Blue},
  pdfcreator={LaTeX via pandoc}}

\title{Introduction to R}
\usepackage{etoolbox}
\makeatletter
\providecommand{\subtitle}[1]{% add subtitle to \maketitle
  \apptocmd{\@title}{\par {\large #1 \par}}{}{}
}
\makeatother
\subtitle{Young Researchers Fellowship Program}
\author{Laboratorio de Investigación para el Desarrollo del Ecuador}
\date{}

\begin{document}
\maketitle

\textbf{Instructor}: Daniel Sánchez, MA

\textbf{Module length}: 15 hours

\textbf{Course level}: Intermediate

\textbf{Prerequisite knowledge}: Basic computer skills, Git, shell.

\textbf{Corequisite knowledge}: \textbf{Introduction to Statistics}
module.

\textbf{GitHub repository}:
https://github.com/laboratoriolide/intro-to-r

\section{Module overview}\label{module-overview}

This module introduces the R programming language basics for data
analysis and statistical computing. This is a companion module to the
\emph{Introduction to Statistics} module, which provides the foundations
of statistics for social science research. The material is designed for
those who aim to use R as a tool for quantitative research methods, such
as statistics, econometrics, and (to a lesser degree), data
science/machine learning.

I assume little prior coding experience: familiarity with basic computer
skills, Git and the command line. Further, the course will be relatively
fast-paced, combining self-study with the DataCamp platform and live
coding. We will cover many important concepts in the language, however,
the focus is not on technical programming but rather on understanding
the language as a tool for statistical analysis.

\section{Evaluation}\label{evaluation}

The final grade will be calculated as follows:

\begin{longtable}[]{@{}ll@{}}
\toprule\noalign{}
Component & Percentage \\
\midrule\noalign{}
\endhead
\bottomrule\noalign{}
\endlastfoot
Attendance & 15\% \\
DataCamp assignments & 50\% \\
Participation & 5\% \\
Task Completion & 30\% \\
\end{longtable}

\subsection{Attendance}\label{attendance}

I do not have any special requirements for attendance other than the
requirements set by the program. Consult the program's regulation
handbook for more information.

\subsection{Assignments}\label{assignments}

There will be weekly DataCamp assignments that reinforce the material
taught in this module as well as that of the companion module
\emph{Intro to Statistics} module. These will be either DataCamp
courses, projects or exercises, and are graded on a pass/fail basis.
Please check the DataCamp platform for the due dates of each assignment.

Apart from the DataCamp assignments, you may need to complete some
self-study or autonomous practice work \textbf{before} each lecture.
This is necessary to make the most of the class time, as programming is
a skill that requires practice. Please see above for all tasks to be
completed before each lecture.

\subsubsection{Tasks}\label{tasks}

Tasks are small exercises or activities that are due before each
lecture. They may be small coding exercises, readings or installations
that you \textbf{must} complete before class. These will be graded on a
pass/fail basis. We may discuss these tasks in class or in the Slack
channel, so it is important that you complete them on time.

\subsubsection{Tentative DataCamp
assignments}\label{tentative-datacamp-assignments}

This is a list of the DataCamp courses that will be assigned as part of
the module. This list may change, so please check the DataCamp platform
for the most up-to-date information.

\begin{itemize}
\tightlist
\item
  Introduction to R for Finance
\item
  Introduction to Importing data in R
\item
  Introduction to the Tidyverse
\item
  Data Manipulation with dplyr
\item
  Introduction to Data Visualization with ggplot2
\item
  Reporting with R Markdown
\item
  Reshaping Data with tidyr (select chapters)
\item
  Intermediate R for Finance
\item
  Cleaning Data in R
\item
  Working with Dates and Times in R
\end{itemize}

\section{Module contents}\label{sec-contents}

The following is a planned outline of the course. This may change
depending on the pace of the class. It is sometimes expected that
students complete tasks before each lecture to make the most of the
class time.

\begin{itemize}
\tightlist
\item
  \textbf{Lecture 1}: Intro

  \begin{itemize}
  \tightlist
  \item
    Introduction to the R language: brief history, advantages and
    applications
  \item
    The development environment: R, RStudio, IDEs, and packages.
  \item
    The RStudio interface: R scripts, file viewer, console, and
    environment.
  \item
    Ensuring reproducibility: filepaths, working adirectories, and
    project organization; using R projects with RStudio.
  \item
    Installing, loading and updating packages, best practices.
  \item
    Basic R syntax: variables, assigners, data types, structures, basic
    and logical operators.
  \item
    Getting help with R: \texttt{?}, \texttt{help()}. Using R
    documentation, CRAN, StackOverflow \& other resources.
  \item
    Data frames: definition and basic manipulation with base R syntax:
    \texttt{head()}, \texttt{tail()}, \texttt{str()},
    \texttt{summary()}. Basic R graphics.
  \item
    Artificial intelligence tools for R coding: ChatGPT, GitHub Copilot,
    Microsoft Copilot.
  \item
    \textbf{Tasks}: Read the full syllabus, installation of R and
    RStudio, cloning of the course's GitHub repository and completion of
    the first DataCamp course, \emph{Introduction to R for Finance}.
  \end{itemize}
\item
  \textbf{Lecture 2}: Introduction to the tidyverse and importing data

  \begin{itemize}
  \tightlist
  \item
    Base R vs the \emph{tidyverse}
  \item
    Introduction to the \emph{tidyverse}: the tidy process, tidy data,
    the pipe (\texttt{\%\textgreater{}\%}) operator
  \item
    Importing data with base R functions, \texttt{readr} \&
    \texttt{haven} from different file formats
  \item
    Importing Excel files and preliminary manipulation with
    \texttt{readxl}
  \item
    Downloading data from R packages, GitHub and the web/URLs
  \item
    Tibbles: the tidy data frame; using \texttt{glimpse()} to understand
    data
  \item
    dplyr: Selecting and renaming columns, pipe workflows,
    \texttt{transmute()}.
  \item
    \textbf{Tasks}: DataCamp course \emph{Introduction to importing data
    in R} and \emph{Introduction to the Tidyverse}. and make sure you
    have access to the module's \texttt{data} folder in the GitHub
    repository.
  \end{itemize}
\item
  \textbf{Lecture 3}: More on data manipulation

  \begin{itemize}
  \tightlist
  \item
    Filtering rows, sorting, grouping, and summarizing data.

    \begin{itemize}
    \tightlist
    \item
      The \texttt{filter()} function for subsetting data frames.
    \item
      The \texttt{group\_by()} and \texttt{summarize()} functions for
      grouping and summarizing data.
    \item
      Leveraging operators: \texttt{==}, \texttt{!=}, \texttt{\%in\%},
      \texttt{\&}, \texttt{\textbar{}}, \texttt{\textless{}},
      \texttt{\textgreater{}}, \texttt{\textless{}=},
      \texttt{\textgreater{}=}, \texttt{between()}.
    \end{itemize}
  \item
    Using \texttt{mutate()} to create new columns: \texttt{if\_else()},
    \texttt{case\_when()}, \texttt{case\_match()}.
  \item
    Others: \texttt{pull()}, \texttt{glance()}, \texttt{slice()},
    \texttt{distinct()}, \texttt{count()}, \texttt{top\_n()},
    \texttt{unique()}, \texttt{arrange()}.
  \item
    Joins, semijoins and antijoins: applications for researchers.
  \item
    The \texttt{bind\_rows()} function for appending data frames.
  \item
    \textbf{Tasks}: DataCamp course \emph{Data Manipulation with dplyr}
    and \emph{Joining Data with dplyr}.
  \end{itemize}
\item
  \textbf{Lecture 4}: Data visualization with ggplot2

  \begin{itemize}
  \tightlist
  \item
    Introduction to the \texttt{ggplot2} package: the grammar of
    graphics.
  \item
    The \texttt{ggplot()} function: aesthetics, geoms, and layers.
  \item
    Scatter plots, line plots, bar plots, histograms, and box plots.
  \item
    Faceting: \texttt{facet\_wrap()} and \texttt{facet\_grid()}.
  \item
    Themes and customization: \texttt{theme()}, \texttt{labs()},
    \texttt{scale\_x\_continuous()}, \texttt{scale\_fill\_manual()}.
  \item
    Saving plots, changing plot size, and exporting to different
    formats.
  \item
    \textbf{Tasks}: DataCamp course \emph{Introduction to Data
    Visualization with ggplot2}.
  \end{itemize}
\item
  \textbf{Lecture 5}: Reshaping data in R

  \begin{itemize}
  \tightlist
  \item
    Understanding long vs.~wide data formats; reshaping data with tidyr.
  \item
    The \texttt{gather()} and \texttt{spread()} equivalents to
    \texttt{pivot\_longer()} and \texttt{pivot\_wider()}.
  \item
    The \texttt{separate()} and \texttt{unite()} functions for splitting
    and combining columns.
  \item
    The \texttt{separate\_rows()} function for splitting rows with
    multiple values.
  \item
    Using the reshape package.
  \item
    Expanding data: \texttt{expand()} and \texttt{complete()} functions.
  \item
    \textbf{Tasks}: DataCamp course \emph{Reshaping Data with tidyr}
    (select chapters).
  \end{itemize}
\item
  \textbf{Lecture 6}: Reporting and reproducibility

  \begin{itemize}
  \tightlist
  \item
    The concept of reproducible research.
  \item
    Introduction to R Markdown: the R Markdown document, YAML header,
    code chunks.
  \item
    Knitting documents: HTML, PDF, Word, and slides.
  \item
    Customizing R Markdown documents: themes, templates, and CSS.
  \item
    Quarto: the next generation of R Markdown.
  \item
    Math formulas: incorporating LaTeX into R Markdown.
  \item
    Time-permitting: \texttt{.Rnw} files and Knitr/Sweave.
  \item
    \textbf{Tasks}: DataCamp course \emph{Reporting with R Markdown}.
    Making sure you have \LaTeX ~installed in your computer, or TinyTeX
    as a package in R\footnote{TinyTeX is a lightweight, portable,
      cross-platform, and easy-to-maintain LaTeX distribution. It is
      available as a package in R, and can be installed with
      \texttt{tinytex::install\_tinytex()}}.
  \end{itemize}
\item
  \textbf{Lecture 7}: Select topics in R programming

  \begin{itemize}
  \tightlist
  \item
    Updating R: finding the R GUI, executing the update, and updating
    RStudio. Potential issues to look out for after updating, selecting
    correct versions of the language and packages in RStudio.
  \item
    Control structures, conditional statements, and logical operators;
    relationships to prepackaged functions.
  \item
    Functions: when to use, how to define functions, arguments, return
    values.
  \item
    The \texttt{apply} family of functions: \texttt{apply()},
    \texttt{lapply()}, \texttt{sapply()}, \texttt{mapply()}.
  \item
    Loops: while and for loops, the \texttt{map} family of functions.
  \item
    \textbf{Tasks}: DataCamp course \emph{Intermediate R for Finance}.
  \end{itemize}
\item
  \textbf{Lecture 8}: Further data cleaning topics

  \begin{itemize}
  \tightlist
  \item
    Handling missing data: base R, \texttt{tidyverse}and package
    approaches.
  \item
    Time and date management: \texttt{lubridate}.
  \item
    String manipulation with \texttt{stringr}.
  \item
    Fuzzy joins: string distance and the \texttt{fuzzyjoin} package.
  \item
    Row-wise and column-wise operations: \texttt{rowwise()},
    \texttt{colwise()}, \texttt{across()}.
  \item
    Using shorthands for \texttt{mutate()} and \texttt{summarize()},
    tidy selection syntax: \texttt{where()}, \texttt{everything()},
    \texttt{c\_across()}, \texttt{c\_where()}, \texttt{starts\_with()},
    etc.
  \item
    \textbf{Tasks}: DataCamp course chapters across \textbf{Cleaning
    Data in R}, \textbf{Working with Dates and Times in R}.
  \end{itemize}
\item
  \textbf{Lecture 9}: Survey data management with R

  \begin{itemize}
  \tightlist
  \item
    Why survey or labelled data might create issues?
  \item
    Using Stata or IBM SPSS data in R: the \texttt{haven} package in
    depth.
  \item
    Developing and using codebooks
  \item
    Applying, removing or changing labels in R: haven's
    \texttt{labelled} class.
  \item
    Working with \texttt{forcats} for factor manipulation.
  \item
    \textbf{Tasks}: TBD, consult the GitHub repository and DataCamp
    platform for updates.
  \end{itemize}
\item
  \textbf{Lecture 10}: High-performance computing in R

  \begin{itemize}
  \tightlist
  \item
    Why high performance? The need for speed in data analysis.
  \item
    Loading very large datasets: \texttt{data.table} and
    \texttt{fread()}.
  \item
    Reading many files at once: \texttt{purrr} and \texttt{map()} with
    \texttt{fread()}.
  \item
    Grouping and summarizing large datasets: \texttt{dplyr} and
    \texttt{data.table} comparison
  \item
    Other alternatives: bigmemoryr, ff, and SQL databases.
  \item
    \textbf{Tasks}: TBD, consult the GitHub repository and DataCamp
    platform for updates.
  \end{itemize}
\item
  \textbf{Other advanced topics}, time-permitting:

  \begin{itemize}
  \tightlist
  \item
    Web scraping: the \texttt{rvest} package.
  \item
    Accessing twitter data: the \texttt{rtweet} package.
  \item
    Accessing financial data: the \texttt{quantmod} package.
  \item
    Time-permiting: APIs, applications with open data.
  \item
    Spatial data analysis with \texttt{sf} and \texttt{sp}.
  \item
    R Setup with VS Code
  \item
    R in the cloud: RStudio Server, RStudio Cloud, and Google Colab.
  \item
    AI chatbot packages: OpenAI tokens, the \texttt{chatgpt} and related
    packages.
  \item
    Using \texttt{usethis} for Git version control and R project
    management.
  \item
    Text analysis.
  \item
    Shiny apps
  \item
    Parallel computing in R
  \end{itemize}
\end{itemize}

\section{Course materials}\label{course-materials}

All course materials will be provided in the course's GitHub repository.
This includes lecture slides, readings, datasets, assignments and any
other relevant material. The repository will be updated regularly, so
please check it often for new material. I recommend using a Git client,
such as GitHub Desktop or GitKraken, to keep your local repository
up-to-date.

\section{Reference material}\label{reference-material}

There is no required textbook for this course, as I will provide slides
for all lectures. However, I recommend the following books for those who
want to delve deeper into the material. These were used as references
for the course.

\begin{itemize}
\item
  \emph{R for Data Science}, Wickham, Çetinkaya-Rundel, and Grolemund
  (2023): Potentially the most famous book on R, it covers the tidyverse
  and data analysis in R. Available also on the web at
  https://r4ds.had.co.nz/, including a Spanish translation,
  https://es.r4ds.hadley.nz/.
\item
  \emph{R Graphics Cookbook}, Chang (2018): A \emph{cookbook} style book
  that covers step-by-step guides on almost all types of visualizations
  you may think of using ggplot2. Note: it does not stress on grammar of
  graphics.
\item
  The Effect, Huntington-Klein (2022): This textbook, while focused on
  causality, describes much of the modern R development environment for
  statistics and econometrics. It also contains Stata and Python code.
\item
  \emph{Using R for Introductory Econometrics} Heiss (2020): This book
  is a great resource for learning R for econometrics, but it will be
  most useful for the Intro to Stats module.
\item
  \href{https://happygitwithr.com/}{\emph{Happy Git and GitHub for the
  useR}}: This book is a great resource for learning Git and GitHub with
  a focus on its use for R programming.
\item
  \emph{R Markdown: The Definitive Guide}, Xie, Allaire, and Grolemund
  (2018): This book is a comprehensive guide to R Markdown, and is a
  great resource for learning how to write reports, papers, and
  presentations with R Markdown.
\item
  \emph{Dynamic Documents with R and knitr}, Xie (2017): The ultimate
  guide to knitr, the package that powers R Markdown. It allows you to
  understand how to fully leverage the power of R for reproducible
  reports.
\item
  \href{https://quarto.org/}{\emph{Quarto website}}: The Quarto website
  is a great resource for learning about it, which is the next
  generation of R Markdown.
\item
  \href{https://book.rfortherestofus.com/}{R for the Rest of Us}
  (\textbf{keyesa?}): A great resource on learning all those things
  about R which aren't about statitsics or data analysis, like package
  development, web scraping, accesing data, creating websites, etc.
\end{itemize}

\subsection{Field-specific references}\label{field-specific-references}

The material that I use in this course is general and can be applied to
any field, having been compiled from a number of different sources. You
may want to consult a field-specific textbook or resource for more
information on how to apply R to your field interest. Below, I'm
including a very short list of resources for some fields (not in any way
exhaustive):

\begin{itemize}
\item
  \href{https://www.bigbookofr.com/}{The Big Book of R}: Contains links
  to resources on R for several fields, topics, and levels of expertise.
  If you want to learn something with the language that I don't cover,
  this is a great place to start.
\item
  \href{https://www.crumplab.com/rstatsforpsych/index.html}{Reproducible
  Statistics for Psychologists with R}
\item
  \href{https://epirhandbook.com/}{The Epidemiologist R Handbook}: A
  great resource for those interested in epidemiology, public health and
  related fields.
\item
  \href{https://www.crimebythenumbers.com/}{Crime by the Numbers: A
  Criminologist's Guide to R}: A great resource for those interested for
  the use of quantitative methods in criminology, criminal justice and
  related fields.
\item
  \href{https://bookdown.org/ripberjt/labbook/}{Lab Guide to
  Quantitative Research Methods in Political Science, Public Policy \&
  Public Administration}: Methods applied to political science, public
  policy and public administration.
\item
  \href{https://gabors-data-analysis.com/}{Data Analysis for Business,
  Economics, and Policy}: R implementations with business and public
  policy/economics applications, with a focus on causal inference, like
  the Effect.
\item
  \href{https://evalsp24.classes.andrewheiss.com/}{Program Evaluation
  for Public Service}: R methods for program evaluation in public
  service, with a focus on causal inference methods. Recommend looking
  at all of Andrew Heiss' resources if interested in applying causal
  inference methods in R.
\item
  \href{https://arcruz0.github.io/libroadp/index.html}{AnalizaR Datos
  Políticos}: A political science-oriented book in Spanish. Great
  resource if looking for datasets and examples for Latin America.
\end{itemize}

\section{Software}\label{software}

This course requires the installation of R and RStudio. Both are free
and open-source software, and are available for Windows, macOS, and
Linux. You will need to install these \textbf{before} the first lecture.

\begin{itemize}
\item
  \textbf{R}: Download the latest version of R from the Comprehensive R
  Archive Network (CRAN) at https://cran.r-project.org/. Follow the
  instructions for your operating system.
\item
  \textbf{RStudio}: Download the latest version of RStudio Desktop from
  \href{https://www.rstudio.com/products/rstudio/download/}{here}.
  Follow the instructions for your operating system.
\end{itemize}

If you are having trouble installing, please post your query in the
Slack channel.

You will also need Git to be able to access the course's GitHub
repository. Follow the instructions from the Git \& GitHub short module
to make sure this software is configured correctly in your computer. As
mentioned, a Git client is recommended to keep your local repository
up-to-date and easily accesible to not miss any updates. For Windows
users, we may use Git Bash sometimes in the course, so make sure you are
familiar with the command line. Mac or Linux users can use the native
terminal.

\section{Keyboard layout}\label{keyboard-layout}

We will routinely need to type symbols like ``/'', ``\textless-'',
``\%\textgreater\%'', and others. Make sure you are comfortable with
your keyboard layout and that you can type these symbols easily. This
may seem trivial, but it is important for the course, as we absolutely
cannot afford to lose time finding symbols on the keyboard. You may need
to change your keyboard layout to the correct language so that the
computer follows the physical layout of your keyboard. For Windows
users, this is easily done by pressing \texttt{Win\ +\ Space} and
selecting the correct layout (see
\href{https://support.microsoft.com/en-us/windows/change-your-keyboard-layout-245c49b8-f856-7fd7-2cf5-41e54c66f5b3}{here})\footnote{Mac
  and Linux Users, sorry, you're on your own here.}.

\section{Communication}\label{communication}

All communications to the instructor should be made through the course's
Slack channel. We hope to respond to questions within 72 hours, but
please be patient if we take longer. I do not monitor email regularly,
so please use Slack for all communications if you need a timely
response.

\section*{References}\label{references}
\addcontentsline{toc}{section}{References}

\phantomsection\label{refs}
\begin{CSLReferences}{1}{0}
\bibitem[\citeproctext]{ref-chang2018r}
Chang, Winston. 2018. \emph{R Graphics Cookbook: Practical Recipes for
Visualizing Data}. O'Reilly Media. \url{https://r-graphics.org/}.

\bibitem[\citeproctext]{ref-heiss20}
Heiss, Florian. 2020. \emph{Using {R} for {Introductory Econometrics}}.
Düsseldorf: Independently published. \url{https://www.urfie.net/}.

\bibitem[\citeproctext]{ref-huntington-klein22}
Huntington-Klein, Nick. 2022. \emph{The Effect: {An} Introduction to
Research Design and Causality}. 1st edition. Boca Raton: {Chapman and
Hall/CRC}. \url{https://theeffectbook.net/}.

\bibitem[\citeproctext]{ref-wickham_etal23}
Wickham, Hadley, Mine Çetinkaya-Rundel, and Garrett Grolemund. 2023.
\emph{R for Data Science}. " O'Reilly Media, Inc.".
\url{https://r4ds.hadley.nz/}.

\bibitem[\citeproctext]{ref-xie2017dynamic}
Xie, Yihui. 2017. \emph{Dynamic Documents with {R} and Knitr}. {Chapman
and Hall/CRC}.
\url{https://duhi23.github.io/Analisis-de-datos/Yihue.pdf}.

\bibitem[\citeproctext]{ref-xie2018r}
Xie, Yihui, Joseph J Allaire, and Garrett Grolemund. 2018. \emph{R
Markdown: {The} Definitive Guide}. {Chapman and Hall/CRC}.
\url{https://bookdown.org/yihui/rmarkdown/}.

\end{CSLReferences}



\end{document}
